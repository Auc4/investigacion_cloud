% ----- PRE DOCUMENTO ----- %
\documentclass[journal]{IEEEtran}


\usepackage[utf8]{inputenc}    
\usepackage[backend=biber,style=ieee]{biblatex}
\usepackage{graphicx} 
\addbibresource{references.bib} 

\title{Security in Cloud}

\author{
    \IEEEauthorblockN{1\textsuperscript{st} Nombre Autor}
    \IEEEauthorblockA{
        \textit{Nombre Escuela} \\
        \textit{Nombre Universidad} \\
        Ciudad, País \\
        email@ejemplo.com
    }
}


% ----- DOCUMENTO ----- %
\begin{document}

\maketitle

\begin{abstract}
    Text...
\end{abstract}

\begin{IEEEkeywords}
    lorem, ipsum
\end{IEEEkeywords}

\section{Introduction}
Text...

% -------------------------------- %

\section{Theoretical Framework}

\subsection{Cloud Computing Essentials}
\label{sec:cloud_essentials}

Cloud computing (CC) is defined as the delivery of computing resources over the internet; including compute power, storage, and databases \cite{OverviewAmazonWeb2025}. This service is based on two main aspects: the on-demand delivery of these assets, and a pay-as-you-go pricing model \cite{OverviewAmazonWeb2025}.

This service has changed how organizations offer their services because with CC a user can access their files from any device, which is running on a remote server rather than being locally stored. This makes it possible to store terabytes of data or stream thousands of movies \cite{cloudflareWhatCloudCloud}. Additionally, organizations can take advantage of CC since it eliminates the need for large investments in on-premise hardware and can reduce energy or maintenance costs \cite{OverviewAmazonWeb2025}.

Moreover, there are some key characteristics of CC:
\begin{itemize}
    \item \textbf{On-demand model:} The customer can increase computing capabilities based on their needs. This is a unilateral process, where users only pay for the computing resources they use \cite{mellNISTDefinitionCloud2011,OverviewAmazonWeb2025}.
    \item \textbf{Access through the internet:} Management tools are accessible via the internet, enabling access from various devices, such as tablets, personal computers, or mobile phones \cite{mellNISTDefinitionCloud2011}.
    \item \textbf{Resource pooling:} The cloud provider groups different resources to serve multiple customer demands. Resources are dynamically allocated to customers, often offering high-level abstract options, such as selecting the country, state, or data center group \cite{mellNISTDefinitionCloud2011}.
    \item \textbf{Speed and Agility:} Resources can be rapidly assigned, allowing services to scale based on demand. This enables customers and project teams to access more resources within minutes rather than weeks \cite{mellNISTDefinitionCloud2011,OverviewAmazonWeb2025}.
\end{itemize}

\subsubsection{Service Models}
\label{sec:service_models}

There are different service models associated with CC. According to \cite{cloudflareWhatCloudCloud}, cloud providers offer different categories of products, each with varying levels of abstraction.

\begin{figure}[ht]
    \includegraphics[width=\columnwidth]{media/cloud_services.png}
    \caption{Cloud Computing Service Models (Adapted from \cite{cloudflareWhatCloudCloud}).}
    \label{fig:service_models}
\end{figure}

As shown in the figure \ref{fig:service_models}, there are multiple service models associated with CC. As defined in \cite{mellNISTDefinitionCloud2011,cloudflareWhatCloudCloud}, these models are known as Infrastructure as a Service (IaaS), Platform as a Service (PaaS) and Software as a Service (SaaS) respectively. According to figure \ref{fig:service_models}, PaaS and SaaS models are built on top of IaaS components because IaaS forms the basis of CC, and each service runs on at least on one physical server.

The general description of these models is:
\begin{itemize}
    \item \textbf{Software as a Service:} This model provides an application that is hosted within a cloud infrastructure. In addition, the customer can access these services from different devices and they do not control the resources of the application; for instance, server configurations, storage allocated, etc \cite{mellNISTDefinitionCloud2011}. 
    \item \textbf{Platform as a service:} In this model, the cloud provider offers the possibility to deploy customer-created applications using programming languages, management services and other tools that support the development process of an application. Nevertheless, the customer does not manage the underlying hardware resources \cite{mellNISTDefinitionCloud2011}. 
    \item \textbf{Infrastructure as a service:} The infrastructure as a Service model provides to the customer fundamental computing resources; such as networks, processing or storage. In this service, the customer can deploy specific software or services depending on their requisites. However, the customer is not able to manage physical infrastructure and security hardware components \cite{mellNISTDefinitionCloud2011}.  
\end{itemize}

\subsubsection{Deployment Models}
\label{sec:deployment_models}

As defined in \cite{cloudflareWhatCloudCloud}, the definition of the deployment models for cloud computing (CC) is related to the specification of the location of these services, and who is responsible for management. 

Different deployment models define how resources are located, managed and consumed: 

\begin{itemize}
    \item \textbf{Private Cloud:} Private cloud consists of using dedicated architecture by a single organization. In addition, the management of these resources is the responsibility of the organization, a third party vendor or a combination or both options taking an on-premise or off-premise solution \cite{mellNISTDefinitionCloud2011}.
    \item \textbf{Public Cloud:} A public cloud offers computing resources to the general public. These resources are owned and managed by a cloud provider, which can be a commercial provider, an organization, or a government entity \cite{mellNISTDefinitionCloud2011}. In addition, the public cloud is used for taking advantage of CC, develop and distribute a service using low-level infrastructure, as seen in IaaS; or a high level platform, as seen in PaaS \cite{OverviewAmazonWeb2025}.
    \item \textbf{Hybrid Cloud:} A hybrid cloud combines public and private cloud environments. Each model has its own independence, but they are connected through standardized processes and systems \cite{mellNISTDefinitionCloud2011}. Additionally, the most common case is to use legacy on-premise infrastructure of an organization, and connect it to the cloud to improve procedures and combine the benefits of each model \cite{OverviewAmazonWeb2025}.
\end{itemize}

Furthermore, the private cloud has different approaches due to its requirements have higher costs than public cloud solutions. According to \cite{stackscaleTypesCloudCloud}, there are multiple solutions for private cloud. First, the \textit{on-premise solutions} are used for physical resources within the organization. Second, the management of the private cloud could be shared with a third party vendor that hosts the infrastructure; this is called \textit{hosted private cloud}. Finally, there is a solution to isolate the resources used within the public cloud to have a control of hardware and network elements; this solution is called as \textit{Virtual Private Cloud} (VPC). 


\subsection{Cloud Security Principles}

As described in \cite{fortinetCloudSecurity}, security in cloud environments refers to the set of technologies and practices used to protect cloud-based environments from threats and vulnerabilities that could impact an organization.

Adopting these techniques enhance multiple aspects of an organization. According to \cite{fortinetCloudSecurity}, adopting cloud security measures can make the work environment safer for employees, ensuring data management and helping to meet regulatory compliances. Consequently, there are various principles that could be used to have a general overview of cloud security.

\subsubsection{CIA Triad Model}

\begin{figure}[ht]
    \centering
    \includegraphics[width=\columnwidth]{media/cia_triad.png}
    \caption{CIA Triad Diagram (Adapted from \cite{geeksforgeeksCiaTriad})}
    \label{fig:cia_triad}
\end{figure}

The figure \ref{fig:cia_triad} shows a general diagram the CIA triad, a foundational model used for developing security systems \cite{fortinetCiaTriad,geeksforgeeksCiaTriad}. Each point within the diagram refers to the pillars of Confidentiality, Integrity and Availability (CIA). 

Each pillar of CIA can be described as follows:
\begin{itemize}
    \item \textbf{Confidentiality:} Ensures that the data is only available for authorized users, keeping it secret and private, even if the breaches of security are intentional or accidental. \cite{geeksforgeeksCiaTriad,fortinetCiaTriad}.
    \item \textbf{Integrity:} Integrity refers to maintain a data consistency, ensuring that is not tampered by unauthorized actors; therefore, the data is reliable, authentic and accurate \cite{fortinetCiaTriad,geeksforgeeksCiaTriad}. 
    \item \textbf{Availability:} The services are accesible when needed, reducing downtime. If the services are not available, they are useless \cite{geeksforgeeksCiaTriad,fortinetCiaTriad}.
\end{itemize}

These model is fundamental for making a general review for ensuring security within cloud environments. As \cite{fortinetCiaTriad} describes, an effective system covers this model, and it is deficient if one of the pillars is not met.

\subsubsection{Shared responsibility Model}
As defined in \cite{sharedCrowdrespon}, the sared responsibility model refers to a framework a compliance framework that is used for establish the responsibilities of customers and Cloud Service Provders (CSPs) within cloud environments.

It is critical to unserdstand that the CSPs don't protect each aspect of cloud resources that are offered \cite{sharedCrowdrespon}.
\subsubsection{Auditability and compliance}


\subsubsection{Multi-tenancy}
% -------------------------------- %

% Estado del arte, análisis, etc.
\section{State of the Art}
Text...

\section{Methodology}
Text...

\section{Results}
Text...

\section{Conclusions}
Text... 

\section{Discussions}
Text...s


% ----- REFERENCIAS -----
\printbibliography


\end{document}
